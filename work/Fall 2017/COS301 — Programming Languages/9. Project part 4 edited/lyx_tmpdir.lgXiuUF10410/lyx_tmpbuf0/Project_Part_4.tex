\batchmode
\makeatletter
\def\input@path{{\string"/Ember Library/Ember satellite projects/personal/work/COS301 — Programming Languages/9. Project part 4 edited/\string"/}}
\makeatother
\documentclass[14pt,english]{extarticle}
\usepackage{fontspec}
\setmainfont[Mapping=tex-text,Numbers=OldStyle]{Computer Modern}
\setsansfont[Scale=0.7,Mapping=tex-text]{Linux Biolinum O}
\setmonofont[Scale=0.7]{Fira Code}
\usepackage{geometry}
\geometry{verbose,tmargin=1in,bmargin=1in,lmargin=1in,rmargin=1in}
\usepackage{setspace}
\setstretch{1.5}

\makeatletter
%%%%%%%%%%%%%%%%%%%%%%%%%%%%%% User specified LaTeX commands.
\usepackage{indentfirst}
\setmonofont[
  Contextuals={Alternate}
]{Fira Code}
\usepackage{hyperref}
\usepackage{url}

\makeatother

\usepackage{listings}
\lstset{basicstyle={\texttt},
columns=fullflexible,
keepspaces=true,
literate={-->}{\texttt{-->}}{1} {->}{\texttt{->}}{1} {==>}{\texttt{==>}}{1} {=>}{\texttt{=>}}{1}}
\usepackage{xunicode}
\usepackage{polyglossia}
\setdefaultlanguage{english}
\begin{document}

\title{Perl 6: Project: Part 4}


\author{Elliot Chandler (Wallace)}


\date{10 October 2017}

\maketitle

\subsubsection*{1. Data types}

Perl 6 includes a large amount of non-primitive types in the language
itself: 33 compound types, 56 ``domain-specific'' types, and 113
exception types. This includes common compound types such as arrays,
lists, and hashes, as well as various other types for a range of applications,
such as for documentation, concurrent programming, and I/O for various
operating systems. Perl 6 includes a large amount of capabilities
within the language core itself. That allows the base language installation
to be quite functional. It means that it is not necessary to install
as many additional libraries to have a practical base of tools \cite{Documentation}.


\subsubsection*{2. Standard library}

Perl 6 does not itself have a specified standard library \emph{per
se}, but the Rakudo Star distribution (the primary way to obtain an
implementation of the language) does include a variety of additional
\emph{modules} (46 of them) \cite{New1}. Rakudo Star includes tools
for performance debugging and profiling, such as \texttt{debugger-ui-commandline},
\texttt{grammar-debugger} and \texttt{Grammar-Profiler-Simple}. It
provides the \texttt{Terminal-ANSIColor} module for creating colored
output from command-line programs, and \texttt{test-mock} and \texttt{Test-META}
for creating unit tests. Rakudo Star provides tools for data interchange
and serialization, in the form of seven modules for working with JSON,
and one for XML, as well as a database interface module (\texttt{DBIlish}).
For graphics, Rakudo Star comes with the \texttt{svg} and \texttt{svg-plot}
modules. It also includes several such as \texttt{perl6-http-easy}
and \texttt{perl6-http-status} for network interoperation. Perl 6
includes a package manager module, also, called \texttt{zef}, which
provides a command-line interface for managing the packages installed
in a system \cite{New1}.


\subsubsection*{3. Semantics of expression evaluation}

Perl 6 makes a distinction between ``statements'' and ``expressions''.
Expressions in Perl 6 are statements that return a value. A program
in Perl 6 semantically represents a list of statements (things for
the computer to do). To make a statement act as an expression, the
\texttt{do} keyword can be used, which will cause the statement given
to represent its value in the given context. Perl 6 generally evaluates
statements in the order in which they are presented in the program
(following the natural control flow), although internally it will
divide up statements to be run concurrently in multiple threads if
the result is the same \cite{Documentation}.


\subsubsection*{4. Type coercion}

By default, Perl 6 when encountering a variable that does not have
a specified type will convert it to the necessary type when it is
used in a context where its current type is not applicable. To override
this behavior, types can be specified for each object, allowing strict
type checking. When a specific type is requested for a variable, assignments
to that variable will not be coerced, and explicit casting is required
\cite{Documentation}.


\subsubsection*{5. Semantics of assignment statements}

Perl 6 has complex assignment semantics, similar to Java, because
it uses containers for some values, but values can also exist without
containers. Containers are object-oriented wrappers around the native
(native to the Perl 6 virtual machine) types. The documentation writes
that ``The assignment operator asks the container on the left to
store the value on its right'' \cite{Documentation}. Because of
this, the actual effect of assigning a value to a name depends on
the type of container that has been created by variable declaration.
In practice, this is largely transparent to the user, and the context
of a reference to a variable will affect whether the container is
used or the value within the container is used. If the user wants
control over this, the language provides some constructs for controlling
these issues \cite{Documentation}, but in my experience the defaults
have generally worked well.

A list can contain containers or values. Because of this, it is not
always possible to assign to list elements, because lists are immutable
and values cannot be changed. However, if the items in a list are
containers, they can be changed because the container itself will
stay the same. That is because the containers themselves do not change;
rather, what they point to changes. Unlike lists, arrays only hold
containers. Consequently, their contents can be changed \cite{Documentation}.

\newpage{}\bibliographystyle{ieeetr}
\nocite{*}
\bibliography{0_Ember_Library_Ember_satellite_projects_person___s_9__Project_part_4_edited_2017sept20-p6bib}



\subsubsection*{Appendix}

The appendices are not included here due to length. Tables of the
composite types, domain-specific types, and exception types are available
in the Perl 6 documentation \cite{Documentation}. A list of modules
included with Rakudo Star is available on request from this author,
based on the information from \cite{New1}.
\end{document}
